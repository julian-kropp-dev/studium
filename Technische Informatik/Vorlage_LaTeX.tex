\documentclass[12pt,a4paper,oneside,ngerman]{article} 
\usepackage[left=3cm,right=3cm,top=2.5cm]{geometry} % Groesse der Seitenraender definieren
\usepackage[utf8]{inputenc} % utf8 encoding
\usepackage{hyperref}
\usepackage{ngerman}
\usepackage{enumerate}
\usepackage{amsmath,amssymb} % Mathe-Formeln und -Ausdruecke
\usepackage{listings} % Code-Ausschnitte einbinden
\usepackage{xcolor} % Eigene Farben definieren
\usepackage{colortbl} % Farben verwenden in Tabellen
\usepackage{multicol} % Mehrspaltigen Text schreiben
\usepackage{caption}
\usepackage{graphicx}
\usepackage{amsmath}
\usepackage{float}

% Titel in Kopfzeilen
\usepackage{fancyhdr}
\pagestyle{fancy}
\setlength{\headheight}{20pt}

% % % % % % % % % % % % % % % % % % % % % % % % % % % % % % 
%Variablen/Befehle -> Mit euren Informationen füllen!
% % % % % % % % % % % % % % % % % % % % % % % % % % % % % % 
\newcommand{\fach}{Grundlagen der Technischen Informatik}
\newcommand{\dokumentenTitel}{Übung YY }
\newcommand{\tutorium}{X }
\newcommand{\memberOne}{Mitglied 0}
\newcommand{\memberTwo}{Mitglied 1}
\newcommand{\memberThree}{Mitglied 2}
\newcommand{\memberFour}{Mitglied 3}
\newcommand{\memberFive}{Mitglied 4}
\newcommand{\group} {ZZ}
% % % % % % % % % % % % % % % % % % % % % % % % % % % % % 

% Kopfzeile auf jeder Seite:
\fancyhead[R]{\dokumentenTitel, \tutorium} % Dokument-Titel
\fancyhead[L]{\memberOne, \memberTwo, \memberThree} % Autorennamen

% % % % % % % % % % % % % % % % % % % % % % % % % % % % % % 
% Hier ist die Kopfzeile und die ganzen Formalia
% % % % % % % % % % % % % % % % % % % % % % % % % % % % % %

\begin{document}
	\thispagestyle{plain} % Keine Kopfzeile auf erster Seite, aber Seitenzahl wird angezeigt
	
	\begin{multicols}{2} % Beginnt zweispaltigen Text fuer Header auf erster Seite
		\hspace{-1cm} % Linken Header-Teil 1cm nach links schieben.
		% Tabelle fuer linke Seite vom Header der ersten Seite
		\begin{tabular}{lll} % Mit l werden die Eintraege linksbuendig
			Gruppe: & \group & \\
			Autoren: & \memberOne & \memberTwo \\ % Zwischen jeder Spalte ein & einfuegen
			& \memberThree & \memberFour \\
			& \memberFive & \\
		\end{tabular}
		
		\columnbreak % Nun beginnt die rechte Seite des Headers
		\hspace{-3cm} % Rechten Header-Teil 1cm nach links schieben.
		% Tabelle fuer rechte Seite vom Header der ersten Seite
		\raggedleft \begin{tabular}{ll}
			Tutorium: &  \ \tutorium \\
			Punkte: &     
			\renewcommand{\arraystretch}{1.2} %Mit diesem Befehl wird die Zeilenhoehe der folgenden Tabelle um 20% erhoeht.
			% Nun kommt eine innere Tabelle in der aeusseren Tabelle, mit der eine Punktetabelle fuer den Tutor erstellt wird:  
			
% % % % % % % % % % % % % % % % % % % % % % % % % % % % % %
% Punktetabelle: Anpassen je nach Aufgabenanzahl :)
% % % % % % % % % % % % % % % % % % % % % % % % % % % % % %
			\begin{tabular}{|p{0.8cm}|p{0.8cm}|p{0.8cm}|p{0.8cm}|} %Spaltenanzahl und breite
				\hline A1.1 & A1.2 & A1.3 & $\sum\limits^{ }$ \\ \hline %obere Zeile
				& & & \\ \hline   %untere Zeile
			\end{tabular}
		\end{tabular}	
	\end{multicols} % Beendet zweispaltigen Text
	
% % % % % % % % % % % % % % % % % % % % % % % % % % % % % % 
% Nun beginnt das eigentliche Dokument:
% % % % % % % % % % % % % % % % % % % % % % % % % % % % % %
	\begin{center}
		\Large{\fach} \\
		\LARGE{\dokumentenTitel} \\
		\small
    \end{center}
       

	\section*{Aufgabe 1.1: Leistungsbewertung}
	Dies ist ein Beispieltext. Ihr könnt Texte, wie Aufgabenstellungen oder Lösungen einfach hier rein schreiben.\\
	So erstellt ihr eine Stichpunktliste:
	\begin{itemize}
	    \item Erster Stichpunkt
	    \item Zweiter Stichpunkt
	\end{itemize}
	
	\subsection*{1. Rechnungen?} \label{ex_1_1}
	\begin{equation}
	    CPI= Gute Frage
	\end{equation}
	
	Oder aber auch:\\
	\begin{equation*}
	    CPI= Gute Frage
	\end{equation*}
	
	Oder:\\
	\begin{align*}
	    3 &= \frac{1}{1-p(I)+\frac{p(5)}{5}} \\
	    1 &= 3\cdot (1-p(I)+\frac{p(I)}{5})\\
	\end{align*}
	
	Im Text könnt ihr Formeln wie $x=2y$ ebenfalls darstellen.\\
	
	Mehr Hilfe dazu hier: \url{https://ftp.rrzn.uni-hannover.de/pub/mirror/tex-archive/macros/latex/required/amsmath/amsldoc.pdf}
	
	\subsection*{2. Fehlt noch was?}
	
	\section*{Aufgabe 1.2: Abbildungen und Tabellen einfügen}
	\paragraph{a)} \textbf{Eritteln Sie den schaltalgebraischen Ausdruck des Schaltnetzes}\\
	So könnt ihr Abbildungen (wie z.b. Abb.~\ref{fig:fechner}) einfügen.
	
	\begin{figure}[H]
	    \centering
	    \includegraphics[width=0.5\textwidth]{Bild.png}
	    \caption{Dies ist eine Bildunterschrift. }
	    \label{fig_abbildung}
	\end{figure}

Und Tabellen (vgl. Tabelle \ref{tab:wwt_xyz}) könnt ihr so erstellen:

\begin{table}[H]
    \centering
    \begin{tabular}{c|c|c||c|c||c}
      & & & D &  & \\
    A & B & C & A $\wedge$ B  & D $\vee$ C & X \\ \hline
    0 & 0 & 0 & & &\\
    0 & 0 & 1 & & &\\
    0 & 1 & 0 & & &\\
    0 & 1 & 1 & & &\\
    1 & 0 & 0 & & &\\
    1 & 0 & 1 & & &\\
    1 & 1 & 0 & & &\\
    1 & 1 & 1 & & &\\
    \end{tabular}
    \caption{Hier kann ein kurze Erläuterung der Tabelle hin}
    \label{tab_wwt_xyz}
\end{table}	
	Ihr könnte Tabellen (Tabelle \ref{tab_wwt_xyz}) und Abbildungen (Abbildung \ref{fig_abbildung}) auch referenzieren.
\end{document}
